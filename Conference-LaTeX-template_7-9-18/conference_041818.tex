\documentclass[conference]{IEEEtran}
\IEEEoverridecommandlockouts
% The preceding line is only needed to identify funding in the first footnote. If that is unneeded, please comment it out.
\usepackage{cite}
\usepackage{amsmath,amssymb,amsfonts}
\usepackage{algorithmic}
\usepackage{graphicx}
\usepackage{textcomp}
\usepackage{xcolor}
\def\BibTeX{{\rm B\kern-.05em{\sc i\kern-.025em b}\kern-.08em
    T\kern-.1667em\lower.7ex\hbox{E}\kern-.125emX}}
\begin{document}

\title{Localization with activity recognition and particle filter\\}

\author{
\IEEEauthorblockN{1\textsuperscript{st} Bergauer Philipp}
\IEEEauthorblockA{\textit{TU Graz} \\
Graz, Austria\\
philipp.bergauer@student.tugraz.at
}
\and
\IEEEauthorblockN{2\textsuperscript{nd} Halici {\"O}mer Faruk}
\IEEEauthorblockA{\textit{TU Graz} \\
Graz, Austria\\
halici@student.tugraz.at}
}

\maketitle

\begin{abstract}
For this course an Activity recognition and an localization application for android were Implemented. The Activity Recognition keeps track if the user is moving or staying in the same Position. It is able to classify if the user is standing, sitting or walking. The localization App can be used calculate where the user is in the 16 building of the Inffeldgasse Graz. Because an Particle Filter was used it is only able to get an accurate Position of the user if the user walked around enough.
\end{abstract}

\section{Introduction}
To keep track of the Activity of the user the Acceleration sensor of the Android phone was used. After a view values of the Acceleration sensor the samples were classified with an k Nearest Neighbor(k-NN) Algorithm. The Algorithm is able to classify the three activities sitting, standing and walking. For Localization the Activity recognition was used to keep track if the user is walking or not walking. With this Information an Particle Filter was used to classify were the user is if he walks arround. The Particle Filter didn't use any additional sensor Informations so it does not have your exact   position when the app is started. The user needs to walk arround a little bit and after that the Particle Filter notice where the Person is in the 16 Inffeldgasse building.
The Source Code can be seen here: https://github.com/PlantinumX/AcitivityMonitoring

\section{Activity recognition}
The mobile phone has many sensors. Some of them can be used to track the activity of a person.  After collecting sensor data the patterns in the data can be retrieved and can be used to classify between activities e.g jogging,running, walking.
\subsection{Tensorflow approach}
Our first approach was to use the popular tensorflow framework from Google. The idea was to train and write the cody in python and export a tensorflow model which can be used by our mobile phone. The trainingsdata was taken from the Wireless Sensor Data Mining group.\cite{1}
The IEEEtran class file is used to format your paper and style the text. All margins, 
column widths, line spaces, and text fonts are prescribed; please do not 
alter them. You may note peculiarities. For example, the head margin
measures proportionately more than is customary. This measurement 
and others are deliberate, using specifications that anticipate your paper 
as one part of the entire proceedings, and not as an independent document. 
Please do not revise any of the current designations.

\subsection{K-NN approach}
Before you begin to format your paper, first write and save the content as a 
separate text file. Complete all content and organizational editing before 
formatting. Please note sections \ref{AA}--\ref{SCM} below for more information on 
proofreading, spelling and grammar.

Keep your text and graphic files separate until after the text has been 
formatted and styled. Do not number text heads---{\LaTeX} will do that 
for you.


\section*{Localization}

\subsection*{approach step counter}
The first approach of the Motion Modell was via an step counter. The Plan was to implement it by calculation the mean value in the motion and calculate the standard deviation and if the user takes a step the step counter increases by one. The Problem here was that every movment the user takes would be recognized as a step and it would not be this accurate. The idea then was to make an Fast Fourier Transformation(FFT) to see how often this movment is made and compare it with an threshold. This was discarded because we already had an Activity recognition that is easy to use.

\subsection*{Motion Model}
For the Motion Model of the Localization application the mentioned Activity recognition with the k-NN was used. The k-NN uses 20 samples for each classifikation. For the Motion Model an set of 30 classifikations were used this took round 1.5seconds. The duration for every 20 samples were measured and if the classifikation said that the probability that the user was walking was the highest the duration for this 20 samples were added to the walking duration. So if the user were walking we would know after 1.5sec for how long he was walking. With this knowledge the steps the user takes were calculated by dividing the duration in seconds by 0.5, because an average person takes 2 steps every second while walking. This stepcount was multiplied by 0,95 to get the distance the user was walking.

\subsection*{Particle Filter}
Please number citations consecutively within brackets \cite{b1}. The 
sentence punctuation follows the bracket \cite{b2}. Refer simply to the reference 
number, as in \cite{b3}---do not use ``Ref. \cite{b3}'' or ``reference \cite{b3}'' except at 
the beginning of a sentence: ``Reference \cite{b3} was the first $\ldots$''

Number footnotes separately in superscripts. Place the actual footnote at 
the bottom of the column in which it was cited. Do not put footnotes in the 
abstract or reference list. Use letters for table footnotes.

Unless there are six authors or more give all authors' names; do not use 
``et al.''. Papers that have not been published, even if they have been 
submitted for publication, should be cited as ``unpublished'' \cite{b4}. Papers 
that have been accepted for publication should be cited as ``in press'' \cite{b5}. 
Capitalize only the first word in a paper title, except for proper nouns and 
element symbols.

For papers published in translation journals, please give the English 
citation first, followed by the original foreign-language citation \cite{b6}.

\subsection*{Localization}
For the Localization Application the two above described Motion Model and Particle Filter were used. To calculate the Position of the user in the room. The Motion Model calculates how long the user walked for and the particle Filter calculates were the Person in the room was.

\subsection*{Optimizations particle filter}
\begin{thebibliography}{00}
\bibitem{b1} G. Eason, B. Noble, and I. N. Sneddon, ``On certain integrals of Lipschitz-Hankel type involving products of Bessel functions,'' Phil. Trans. Roy. Soc. London, vol. A247, pp. 529--551, April 1955.
\bibitem{b2} J. Clerk Maxwell, A Treatise on Electricity and Magnetism, 3rd ed., vol. 2. Oxford: Clarendon, 1892, pp.68--73.
\bibitem{b3} I. S. Jacobs and C. P. Bean, ``Fine particles, thin films and exchange anisotropy,'' in Magnetism, vol. III, G. T. Rado and H. Suhl, Eds. New York: Academic, 1963, pp. 271--350.
\bibitem{b4} K. Elissa, ``Title of paper if known,'' unpublished.
\bibitem{b5} R. Nicole, ``Title of paper with only first word capitalized,'' J. Name Stand. Abbrev., in press.
\bibitem{b6} Y. Yorozu, M. Hirano, K. Oka, and Y. Tagawa, ``Electron spectroscopy studies on magneto-optical media and plastic substrate interface,'' IEEE Transl. J. Magn. Japan, vol. 2, pp. 740--741, August 1987 [Digests 9th Annual Conf. Magnetics Japan, p. 301, 1982].
\bibitem{b7} M. Young, The Technical Writer's Handbook. Mill Valley, CA: University Science, 1989.
\end{thebibliography}
\vspace{12pt}
\color{red}
IEEE conference templates contain guidance text for composing and formatting conference papers. Please ensure that all template text is removed from your conference paper prior to submission to the conference. Failure to remove the template text from your paper may result in your paper not being published.

\end{document}
