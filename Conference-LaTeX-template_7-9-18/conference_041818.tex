\documentclass[conference]{IEEEtran}
\IEEEoverridecommandlockouts
% The preceding line is only needed to identify funding in the first footnote. If that is unneeded, please comment it out.
\usepackage{cite}
\usepackage{amsmath,amssymb,amsfonts}
\usepackage{algorithmic}
\usepackage{graphicx}
\usepackage{textcomp}
\usepackage{xcolor}
\def\BibTeX{{\rm B\kern-.05em{\sc i\kern-.025em b}\kern-.08em
    T\kern-.1667em\lower.7ex\hbox{E}\kern-.125emX}}
\begin{document}

\title{Localization with activity recognition and particle filter\\}

\author{
\IEEEauthorblockN{1\textsuperscript{st} Bergauer Philipp}
\IEEEauthorblockA{\textit{TU Graz} \\
Graz, Austria\\
philipp.bergauer@student.tugraz.at
}
\and
\IEEEauthorblockN{2\textsuperscript{nd} Halici {\"O}mer Faruk}
\IEEEauthorblockA{\textit{TU Graz} \\
Graz, Austria\\
halici@student.tugraz.at}
}

\maketitle

\begin{abstract}
This document is a model and instructions for \LaTeX.
This and the IEEEtran.cls file define the components of your paper [title, text, heads, etc.]. *CRITICAL: Do Not Use Symbols, Special Characters, Footnotes, 
or Math in Paper Title or Abstract.
\end{abstract}

\begin{IEEEkeywords}
component, formatting, style, styling, insert
\end{IEEEkeywords}

\section{Introduction}
The Localization was splited into two parts, the first part was an Activity recognition and the second part was the localization Algorithm with the Particle Filter. The Activity recognition was implemented with an k-Nearest Neighbor(k-NN) algorithm. It was used to classify if the user is moving or not. 

\section{Activity recognition}
The mobile phone has many sensors. Some of them can be used to track the activity of a person.  After collecting sensor data the patterns in the data can be retrieved and can be used to classify between activities e.g jogging,running, walking.
\subsection{Tensorflow approach for Convolutional Neural Network}
Our first approach was to use the popular tensorflow framework from Google. The idea was to train and write the cody in python and export a tensorflow model which can be used by our mobile phone. The trainingsdata was taken from the Wireless Sensor Data Mining group\cite{b1}. We decided to use a CNN network because it can be used to analyze interesting fatures in the data set. The error rate for the trainings data was very low. Only 15 \% were missclassified. The export to the android device worked too but in the end it did not work as accurate enough. The problem was that our dataset and our own sensor data generated from our mobile phones were to different so the model missclassified most of the activites wrong. Afterwards we tried to After days of debugging and fixing errors we came to the conclusion to go along with the K-NN approach\cite{b2}.
\subsection{K-NN approach}
First of all the K-NN algorithm is one of the simplest and most used classification algorithms. The classification of an object/class uses the majority vote of its neighbours. There are many decisions to be made before implementing the K-NN. First of all which K should we take. The K stands for the number of neighbours to be taken in account. It should be an odd number. In our case it was 21. The calculation of the distance is another point to think about. We decided to use the euclidean distance for our K-NN. The formula of the euclidean distance:
\begin{equation}
D(w_i,v_i) = \sqrt{\sum{(w_i - v_i)^2}}
\end{equation}
The euclidean distance is easy to program and is very efficient. 
\subsection{Feature extraction}
Afterwards we thought about the features to use for the data set and the window size. We took the same features as for our CNN.The mean for each axis of the accelerometer values, the max peaks in each axis,the min peaks in each axis and the variances of each axis. 
In addition one record contains 20 samples too. The sample rate can be adjusted in android so we took the sampling rate $SENSOR\_DELAY\_FASTEST$. According to its name it should be the fastest one. The windows size are 20 samples. The sampling rate is dependable on the hardware in the mobile phone, so it is hard to give a window for 20 samples in seconds but it should be approximately 50ms. 
\subsection{Data generation}
The reference dataset was  taken with the SensorHandler. These class takes the samples and writed them in a txt file. In our main activity we can click on the generate data button and generate data for our classificitation as shown in 
\ref{fig:mainmenuactivity}. \\ 
\begin{figure}
\includegraphics[height = 7.5cm,width = 5cm]{Images/MainActivity.jpeg}
\centering
\caption{MainMenuActivity of the application}
\label{fig:mainmenuactivity}
\end{figure}\\
With our own dataset and the extracted features we could classify. We differentiated between 3 classes:Walking,Standing and Sitting. It should be explicitly noted that our activity recognition works with the phone in the pocket. 
\subsection{Classifier}
\begin{figure}
\includegraphics[height = 7.5cm,width = 5cm]{Images/AcitivityRecognition.jpeg}
\centering
\caption{Activityrecognition with accelerometer values }
\label{fig:classifier}
\end{figure}
As seen in \ref{fig:classifier} the K-NN algorithm classified sitting right with an accuracy about 80 \%. The classifier is very fast because we use the fastest sampling rate possible. The best classification are walking and standing because the accelerometer values are much apparent to our classifier.
Before you begin to format your paper, first write and save the content as a 
separate text file. Complete all content and organizational editing before 
formatting. Please note sections \ref{AA}--\ref{SCM} below for more information on 
proofreading, spelling and grammar.

Keep your text and graphic files separate until after the text has been 
formatted and styled. Do not number text heads---{\LaTeX} will do that 
for you.


\section*{Localization}

The preferred spelling of the word ``acknowledgment'' in America is without 
an ``e'' after the ``g''. Avoid the stilted expression ``one of us (R. B. 
G.) thanks $\ldots$''. Instead, try ``R. B. G. thanks$\ldots$''. Put sponsor 
acknowledgments in the unnumbered footnote on the first page.

\subsection*{Particle Filter}

Please number citations consecutively within brackets \cite{b1}. The 
sentence punctuation follows the bracket \cite{b2}. Refer simply to the reference 
number, as in \cite{b3}---do not use ``Ref. \cite{b3}'' or ``reference \cite{b3}'' except at 
the beginning of a sentence: ``Reference \cite{b3} was the first $\ldots$''

Number footnotes separately in superscripts. Place the actual footnote at 
the bottom of the column in which it was cited. Do not put footnotes in the 
abstract or reference list. Use letters for table footnotes.

Unless there are six authors or more give all authors' names; do not use 
``et al.''. Papers that have not been published, even if they have been 
submitted for publication, should be cited as ``unpublished'' \cite{b4}. Papers 
that have been accepted for publication should be cited as ``in press'' \cite{b5}. 
Capitalize only the first word in a paper title, except for proper nouns and 
element symbols.

For papers published in translation journals, please give the English 
citation first, followed by the original foreign-language citation \cite{b6}.
\subsection*{Optimizations particle filter}
\begin{thebibliography}{00}
\bibitem{b1} Jennifer R. Kwapisz, Gary M. Weiss, Samuel A. Moore, ``Activity Recognition using Cell Phone Accelerometers,''  Department of Computer and Information Science  Fordham University  441 East Fordham Road  Bronx, NY 10458.
\bibitem{b2}https://aqibsaeed.github.io/2016-11-04-human-activity-recognition-cnn/
\bibitem{b3} I. S. Jacobs and C. P. Bean, ``Fine particles, thin films and exchange anisotropy,'' in Magnetism, vol. III, G. T. Rado and H. Suhl, Eds. New York: Academic, 1963, pp. 271--350.
\bibitem{b4} K. Elissa, ``Title of paper if known,'' unpublished.
\bibitem{b5} R. Nicole, ``Title of paper with only first word capitalized,'' J. Name Stand. Abbrev., in press.
\bibitem{b6} Y. Yorozu, M. Hirano, K. Oka, and Y. Tagawa, ``Electron spectroscopy studies on magneto-optical media and plastic substrate interface,'' IEEE Transl. J. Magn. Japan, vol. 2, pp. 740--741, August 1987 [Digests 9th Annual Conf. Magnetics Japan, p. 301, 1982].
\bibitem{b7} M. Young, The Technical Writer's Handbook. Mill Valley, CA: University Science, 1989.
\end{thebibliography}
\vspace{12pt}
\color{red}
IEEE conference templates contain guidance text for composing and formatting conference papers. Please ensure that all template text is removed from your conference paper prior to submission to the conference. Failure to remove the template text from your paper may result in your paper not being published.

\end{document}
